%%%%%%%%%%%%%%%%%%%%%%%%%%%%%%%%%%%%%%%%%
% Homework Assignment Article
% LaTeX Template
% Version 1.3.5r (2018-02-16)
%
% This template has been downloaded from:
% /cl.uni-heidelberg.de/~zimmermann/
%
% Original author:
% Victor Zimmermann (zimmermann@cl.uni-heidelberg.de)
%
% License:
% CC BY-SA 4.0 (https://creativecommons.org/licenses/by-sa/4.0/)
%
%%%%%%%%%%%%%%%%%%%%%%%%%%%%%%%%%%%%%%%%%

%----------------------------------------------------------------------------------------

\documentclass[a4paper,12pt]{article} % Uses article class in A4 format

%----------------------------------------------------------------------------------------
%	FORMATTING
%----------------------------------------------------------------------------------------

\setlength{\parskip}{0pt}
\setlength{\parindent}{0pt}
\setlength{\voffset}{-15pt}

%----------------------------------------------------------------------------------------
%	PACKAGES AND OTHER DOCUMENT CONFIGURATIONS
%----------------------------------------------------------------------------------------

\usepackage[a4paper, margin=2.5cm]{geometry} % Sets margin to 2.5cm for A4 Paper
\usepackage[onehalfspacing]{setspace} % Sets Spacing to 1.5
%\linespread{2}

\usepackage[T2A]{fontenc} % Use European encoding
\usepackage[utf8]{inputenc} % Use UTF-8 encoding
%\usepackage{lmodern}
%\usepackage{charter} % Use the Charter font
%\usepackage{microtype} % Slightly tweak font spacing for aesthetics

\usepackage[english, russian]{babel} % Language hyphenation and typographical rules

\usepackage{amsthm, amsmath, amssymb} % Mathematical typesetting
\usepackage{marvosym, wasysym} % More symbols
\usepackage{float} % Improved interface for floating objects
\usepackage[final, colorlinks = true, 
            linkcolor = black, 
            citecolor = black,
            urlcolor = black]{hyperref} % For hyperlinks in the PDF
\usepackage{graphicx, multicol} % Enhanced support for graphics
\usepackage{xcolor} % Driver-independent color extensions
\usepackage{rotating} % Rotation tools
\usepackage{listings, style/lstlisting} % Environment for non-formatted code, !uses style file!
\usepackage{pseudocode} % Environment for specifying algorithms in a natural way
\usepackage{style/avm} % Environment for f-structures, !uses style file!
\usepackage{booktabs} % Enhances quality of tables

\usepackage{tikz-qtree} % Easy tree drawing tool
\tikzset{every tree node/.style={align=center,anchor=north},
         level distance=2cm} % Configuration for q-trees
\usepackage{style/btree} % Configuration for b-trees and b+-trees, !uses style file!

\usepackage{titlesec} % Allows customization of titles
%\renewcommand\thesection{\arabic{section}.} % Arabic numerals for the sections
\titleformat{\section}{\large}{\thesection}{1em}{}
%\renewcommand\thesubsection{\alph{subsection})} % Alphabetic numerals for subsections
\titleformat{\subsection}{\large}{\thesubsection}{1em}{}
%\renewcommand\thesubsubsection{\roman{subsubsection}.} % Roman numbering for subsubsections
\titleformat{\subsubsection}{\large}{\thesubsubsection}{1em}{}

\usepackage[all]{nowidow} % Removes widows

\usepackage{csquotes} % Context sensitive quotation facilities

\usepackage[ddmmyyyy]{datetime} % Uses YEAR-MONTH-DAY format for dates
\renewcommand{\dateseparator}{.} % Sets dateseparator to '-'

\usepackage{fancyhdr} % Headers and footers
\pagestyle{fancy} % All pages have headers and footers
\fancyhead{}\renewcommand{\headrulewidth}{0pt} % Blank out the default header
\fancyfoot[L]{\textsc{}} % Custom footer text
\fancyfoot[C]{} % Custom footer text
\fancyfoot[R]{\thepage} % Custom footer text

\newcommand{\note}[1]{\marginpar{\scriptsize \textcolor{red}{#1}}} % Enables comments in red on margin

\usepackage{microtype} % Slightly tweak font spacing for aesthetics
\usepackage{amsthm, amssymb, amsmath, amsfonts, nccmath}
\usepackage{nicefrac}
\usepackage{float} % Improved interface for floating objects
\usepackage{graphicx, multicol} % Enhanced support for graphics
\usepackage{pdfrender,xcolor}
%\usepackage{breqn}
\usepackage{mathtools}
\usepackage{tikz}
\usepackage{marvosym, wasysym} % More symbols
\usepackage{rotating} % Rotation tools
\usepackage{censor}

\usepackage{algorithm}
\usepackage{algpseudocode}

\DeclareMathOperator{\cov}{cov}
\DeclareMathOperator{\med}{med}
\DeclareMathOperator{\sign}{sign}
%----------------------------------------------------------------------------------------

\begin{document}

%----------------------------------------------------------------------------------------
%	TITLE SECTION
%----------------------------------------------------------------------------------------

\large
\begin{center}
    Санкт-Петербургский политехнический университет\\
    Высшая школа прикладной математики и\\вычислительной физики, ИПММ\\
    \vspace{5em}
    Направление подготовки\\
    01.03.02 «Прикладная математика и информатика»\\
    \vspace{3em}
    Отчет по лабораторной работе №8\\
    по дисциплине «Математическая статистика»
    \vspace{15em}
\end{center}
Выполнил студент гр. 3630102/80201\\
Кирпиченко С. Р.\\
Руководитель\\
Баженов А. Н.
\vspace{7em}
\begin{center}
    Санкт-Петербург\\
    2021
\end{center}
\thispagestyle{empty}
\newpage
\tableofcontents
\addtocontents{toc}{~\hfill\textbf{Страница}\par}
\newpage
\listoffigures
\addtocontents{lof}{~\hfill\textbf{Страница}\par}
\newpage
\listoftables
\addtocontents{lot}{~\hfill\textbf{Страница}\par}
\thispagestyle{empty}
\newpage
%----------------------------------------------------------------------------------------
%	TITLE SECTION
%----------------------------------------------------------------------------------------
\section{Постановка задачи}
Провести дисперсионный анализ с применением критерия Фишера по
данным регистраторов для одного сигнала. Определить области однородности
сигнала, переходные области, шум/фон.
\section{Теория}
По гистограмме входного сигнала можно разметить его области:
\begin{enumerate}
    \item Столбец с самым большим значением описывает точки, отвечающие за шум (фоновый сигнал);
    \item Второй по величине столбец описывает точки, отвечающие непосредственно за сигнал;
    \item В остальные столбцы попадают точки, описывающие переходное состояние между сигналом и фоном.
\end{enumerate}
Области однородности определяются с помощью критерия Фишера: однородные области определяются значениями, близкими к 1, переходные - большими.
Значение критерия Фишера
$$F=\dfrac{\sigma^2_{InterGroup}}{\sigma^2_{IntaGroup}}$$
определено внутригрупповой дисперсией $\sigma^2_{InterGroup}$ и межгрупповой $\sigma^2_{IntaGroup}$.
$$\sigma^2_{InterGroup}=\dfrac{1}{k}=\sum\limits_{i=1}^k\dfrac{\sum_{j=1}^n(x_{ij}-\overline{x})^2}{k-1}$$
$$\sigma^2_{IntaGroup}=k\dfrac{\sum^k_{i=1}(\overline{x}_i+\overline{X})^2}{k-1},$$
где $\overline{x}$ - среднее для части выборки, $k$ - количество частей выборки, $n$ - количество элементов в рассматриваемой части выборки. $\overline{x}_i$ - среднее значение подвыборок, $\overline{X}$ - среднее значение этих средних значений.

Перед определением областей однородности необходимо устранить выбросы, сгладив сигнал медианным фильтром.
\section{Реализация}
Лабораторная работа выполнена на языке Python 3.9 с использованием библиотек numpy, scipy, matplotlib, seaborn.
\section{Результаты}
Для исследования был выбран сигнал под номером 228. 
\begin{figure}[H]
    \centering
    \includegraphics[width=15cm]{signal228.pdf}
    \caption{Внешний вид входного сигнала}
    \label{fig:imag}
\end{figure}
\begin{figure}[H]
    \centering
    \includegraphics[width=15cm]{histogram.pdf}
    \caption{Гистограмма входного сигнала}
    \label{fig:hist}
\end{figure}
Для сглаживания сигнала медиана считалась по маске из 5 элементов.
\begin{figure}[H]
    \centering
    \includegraphics[width=14cm]{signal228smoothed.pdf}
    \caption{Сглаженный сигнал}
    \label{fig:smooth}
\end{figure}
\begin{figure}[H]
    \centering
    \includegraphics[width=15cm]{areas.pdf}
    \caption{Сигнал с размеченными областями}
    \label{fig:areas}
\end{figure}
\begin{table}[H]
    \centering
    \begin{tabular}{|c|c|c|c|}
        \hline
         Промежуток&Тип&Число разбиений& Критерий Фишера \\
         \hline
         1&Шум&7&0.42\\
         \hline
         2&Переход&4&15.37\\
         \hline
         3&Сигнал&4&0.13\\
         \hline
         4&Переход&4&16.88\\
         \hline
         5&Шум&7&2.33\\
         \hline
    \end{tabular}
    \caption{Характеристики выделенных областей}
    \label{tab:tab1}
\end{table}
\section{Обсуждение}
Входные данные были разбиты на следующие области: фоновый шум (2 области), 2 перехода и область сигнала.

Области фонового шума и сигнала однородны - критерий Фишера приблизительно равен 1. На переходах значение критерия сильно больше 1 - это области неоднородности.
\section*{Примечание}
С исходным кодом работы и данного отчета можно ознакомиться в репозитории\;\url{https://github.com/Stasychbr/MatStat}
\end{document}
