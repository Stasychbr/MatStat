%%%%%%%%%%%%%%%%%%%%%%%%%%%%%%%%%%%%%%%%%
% Homework Assignment Article
% LaTeX Template
% Version 1.3.5r (2018-02-16)
%
% This template has been downloaded from:
% /cl.uni-heidelberg.de/~zimmermann/
%
% Original author:
% Victor Zimmermann (zimmermann@cl.uni-heidelberg.de)
%
% License:
% CC BY-SA 4.0 (https://creativecommons.org/licenses/by-sa/4.0/)
%
%%%%%%%%%%%%%%%%%%%%%%%%%%%%%%%%%%%%%%%%%

%----------------------------------------------------------------------------------------

\documentclass[a4paper,12pt]{article} % Uses article class in A4 format

%----------------------------------------------------------------------------------------
%	FORMATTING
%----------------------------------------------------------------------------------------

\setlength{\parskip}{0pt}
\setlength{\parindent}{0pt}
\setlength{\voffset}{-15pt}

%----------------------------------------------------------------------------------------
%	PACKAGES AND OTHER DOCUMENT CONFIGURATIONS
%----------------------------------------------------------------------------------------

\usepackage[a4paper, margin=2.5cm]{geometry} % Sets margin to 2.5cm for A4 Paper
\usepackage[onehalfspacing]{setspace} % Sets Spacing to 1.5
%\linespread{2}

\usepackage[T2A]{fontenc} % Use European encoding
\usepackage[utf8]{inputenc} % Use UTF-8 encoding
%\usepackage{lmodern}
%\usepackage{charter} % Use the Charter font
%\usepackage{microtype} % Slightly tweak font spacing for aesthetics

\usepackage[english, russian]{babel} % Language hyphenation and typographical rules

\usepackage{amsthm, amsmath, amssymb} % Mathematical typesetting
\usepackage{marvosym, wasysym} % More symbols
\usepackage{float} % Improved interface for floating objects
\usepackage[final, colorlinks = true, 
            linkcolor = black, 
            citecolor = black,
            urlcolor = black]{hyperref} % For hyperlinks in the PDF
\usepackage{graphicx, multicol} % Enhanced support for graphics
\usepackage{xcolor} % Driver-independent color extensions
\usepackage{rotating} % Rotation tools
\usepackage{listings, style/lstlisting} % Environment for non-formatted code, !uses style file!
\usepackage{pseudocode} % Environment for specifying algorithms in a natural way
\usepackage{style/avm} % Environment for f-structures, !uses style file!
\usepackage{booktabs} % Enhances quality of tables

\usepackage{tikz-qtree} % Easy tree drawing tool
\tikzset{every tree node/.style={align=center,anchor=north},
         level distance=2cm} % Configuration for q-trees
\usepackage{style/btree} % Configuration for b-trees and b+-trees, !uses style file!

\usepackage{titlesec} % Allows customization of titles
%\renewcommand\thesection{\arabic{section}.} % Arabic numerals for the sections
\titleformat{\section}{\large}{\thesection}{1em}{}
%\renewcommand\thesubsection{\alph{subsection})} % Alphabetic numerals for subsections
\titleformat{\subsection}{\large}{\thesubsection}{1em}{}
%\renewcommand\thesubsubsection{\roman{subsubsection}.} % Roman numbering for subsubsections
\titleformat{\subsubsection}{\large}{\thesubsubsection}{1em}{}

\usepackage[all]{nowidow} % Removes widows

\usepackage{csquotes} % Context sensitive quotation facilities

\usepackage[ddmmyyyy]{datetime} % Uses YEAR-MONTH-DAY format for dates
\renewcommand{\dateseparator}{.} % Sets dateseparator to '-'

\usepackage{fancyhdr} % Headers and footers
\pagestyle{fancy} % All pages have headers and footers
\fancyhead{}\renewcommand{\headrulewidth}{0pt} % Blank out the default header
\fancyfoot[L]{\textsc{}} % Custom footer text
\fancyfoot[C]{} % Custom footer text
\fancyfoot[R]{\thepage} % Custom footer text

\newcommand{\note}[1]{\marginpar{\scriptsize \textcolor{red}{#1}}} % Enables comments in red on margin

\usepackage{microtype} % Slightly tweak font spacing for aesthetics
\usepackage{amsthm, amssymb, amsmath, amsfonts, nccmath}
\usepackage{nicefrac}
\usepackage{float} % Improved interface for floating objects
\usepackage{graphicx, multicol} % Enhanced support for graphics
\usepackage{pdfrender,xcolor}
%\usepackage{breqn}
\usepackage{mathtools}
\usepackage{tikz}
\usepackage{marvosym, wasysym} % More symbols
\usepackage{rotating} % Rotation tools
\usepackage{censor}

\usepackage{algorithm}
\usepackage{algpseudocode}
%----------------------------------------------------------------------------------------

\begin{document}

%----------------------------------------------------------------------------------------
%	TITLE SECTION
%----------------------------------------------------------------------------------------

\large
\begin{center}
    Санкт-Петербургский политехнический университет\\
    Высшая школа прикладной математики и\\вычислительной физики, ИПММ\\
    \vspace{5em}
    Направление подготовки\\
    01.03.02 «Прикладная математика и информатика»\\
    \vspace{3em}
    Отчет по лабораторным работам №3\\
    по дисциплине «Математическая статистика»
    \vspace{15em}
\end{center}
Выполнил студент гр. 3630102/80201\\
Кирпиченко С. Р.\\
Руководитель\\
Баженов А. Н.
\vspace{7em}
\begin{center}
    Санкт-Петербург\\
    2021
\end{center}
\thispagestyle{empty}
\newpage
\tableofcontents
\addtocontents{toc}{~\hfill\textbf{Страница}\par}
\newpage
\listoffigures
\addtocontents{lof}{~\hfill\textbf{Страница}\par}
\newpage
\listoftables
\addtocontents{lot}{~\hfill\textbf{Страница}\par}
\thispagestyle{empty}
\newpage
%----------------------------------------------------------------------------------------
%	TITLE SECTION
%----------------------------------------------------------------------------------------
\section{Постановка задачи}
Для 5 распределений:
\begin{itemize}
    \item Нормальное распределение $N(x, 0, 1)$
    \item Распределение Коши $C(x, 0, 1)$
    \item Распределение Лапласа $L(x, 0, \frac{1}{\sqrt{2}})$
    \item Распределение Пуассона $P(k, 10)$
    \item Равномерное распределение $U(x,-\sqrt{3},\sqrt{3})$
\end{itemize}
Сгенерировать выборки размером 20 и 100 элементов.
Построить для них боксплоты Тьюки.
Для каждого распределения определить долю выбросов экспериментально (усредняя долю выбросов по 1000 выборок) и сравнить с результатами, полученными теоретически.
\section{Теория}
\subsection{Определение}
	\noindent Боксплот (англ. box plot) — график, использующийся в описательной статистике, компактно изображающий одномерное распределение вероятностей
	
	\subsection{Описание}
	\noindent Такой вид диаграммы в удобной форме показывает медиану, нижний и верхний квартили и выбросы. Несколько таких ящиков можно нарисовать бок о бок, чтобы визуально сравнивать одно распределение с другим; их можно располагать как горизонтально, так и вертикально. Расстояния между различными частями ящика позволяют определить степень разброса (дисперсии) и асимметрии данных и выявить выбросы.
	
	\subsection{Построение}
	\noindent Границами ящика служат первый и третий квартили, линия в середине ящика — медиана. Концы усов — края статистически значимой выборки (без выбросов). Длину «усов» определяют разность первого квартиля и полутора межквартильных расстояний и сумма третьего квартиля и полутора межквартильных расстояний. Формула имеет вид
	\begin{equation}
	\label{mouns}
	{X_1 = Q_1-} \frac{3}{2}{(Q_3 - Q_1)},   {X_2 = Q_3+} \frac{3}{2}{(Q_3 - Q_1)}
	\end{equation}
	где $X_1$ — нижняя граница уса, $X_2$ — верхняя граница уса, $Q_1$ — первый квартиль, $Q_3$ — третий квартиль. Данные, выходящие за границы усов (выбросы), отображаются на графике в виде маленьких кружков.
	
	
\subsection{Теоретическая вероятность выбросов}
	\noindent Встроенными средствами языка программирования Python в среде разработки PyCharm можно вычислить теоретические первый и третий квартили распределений ($Q_1^T$ и $Q_3^T$ соответственно). По формуле \eqref{mouns} можно вычислить теоретические нижнюю и верхнюю границы уса ($X_1^T$ и $X_2^T$ соответственно). Выбросами считаются величины x, такие что: 
	\begin{equation}
		\left[
		\begin{gathered}
		x < X_1^T \\
		x > X_2^T \\
		\end{gathered}
		\right.
	\end{equation}
	Теоретическая вероятность выбросов для непрерывных распределений
	\begin{equation}
		P_B^T = P(x<X_1^T) + P(x>X_2^T)=F(X_1^T) + (1-F(X_2^T))
	\end{equation}
	где $F(X)=P(x\leq{X})$ - функция распределения.
	Теоретическая вероятность выбросов для дискретных распределений
	\begin{equation}
		P_B^T = P(x<X_1^T)+P(x>x_2^T)=(F(X_1^T)-P(x=X_1^T))+(1-F(X_2^T))
	\end{equation}
	где $F(X) = P(x\leq{X})$ - функция распределения
\section{Реализация}
Лабораторная работа выполнена на языке Python 3.9 с использованием библиотек numpy, scipy, matplotlib, seaborn.
\section{Результаты}
\subsection{Боксплот Тьюки}
\begin{figure}[H]
    \centering
    \includegraphics[width = 15 cm]{normalBox.pdf}
    \caption{Нормальное распределение}
    \label{fig:norm}
\end{figure}
\begin{figure}[H]
    \centering
    \includegraphics[width = 15 cm]{koshiBox.pdf}
    \caption{Распределение Коши}
    \label{fig:cauchy}
\end{figure}
\begin{figure}[H]
    \centering
    \includegraphics[width = 15 cm]{laplaceBox.pdf}
    \caption{Распределение Лапласа}
    \label{fig:laplace}
\end{figure}
\begin{figure}[H]
    \centering
    \includegraphics[width = 15 cm]{poissonBox.pdf}
    \caption{Распределение Пуассона}
    \label{fig:poisson}
\end{figure}
\begin{figure}[H]
    \centering
    \includegraphics[width = 15 cm]{uniformBox.pdf}
    \caption{Равномерное распределение}
    \label{fig:uniform}
\end{figure}
\subsection{Доля выбросов}
\noindent Округление доли выбросов:\\
Выборка случайна, поэтому в качестве оценки рассеяния можно взять дисперсию пуассоновского потока:  $D_n \approx \sqrt{n}$\\
Доля $p_n = \frac{D_n}{n}=\frac{1}{\sqrt{n}}$\\
Доля $n=20: p_n=\frac{1}{\sqrt{20}}$ - примерно 0.2 или 20\% \\
Для $n=100: p_n=\frac{1}{\sqrt{100}}$ - примерно 0.1 или 10\% \\
Исходя из этого можно решить, сколько знаков оставлять в доле выброса.
\begin{table}[H]
\centering
\begin{tabular}{|c|c|c|}
    \hline
     Распределение & Размер выборки & Доля выбросов \\
     \hline
     Нормальное & 20 & 0.0198\\
     \hline
     Нормальное & 100 & 0.00919\\
     \hline
     Коши & 20 & 0.1488 \\
     \hline
     Коши & 100 & 0.1521\\
     \hline
     Лапласа & 20 & 0.0652\\
     \hline
     Лапласа & 100 & 0.06295\\
     \hline
     Пуассона & 20 & 0.0265 \\
     \hline
     Пуассона & 100 & 0.01527 \\
     \hline
     Равномерное & 20 & 0 \\
     \hline
     Равномерное & 100 & 0\\
     \hline
\end{tabular}
		\caption{Теоретическая вероятность выбросов}
		\label{tab:normal}
	\end{table}
\subsection{Теоретическая вероятность выбросов}
	\begin{table}[H]
		\centering
		\begin{tabular}[t]{|c|c|c|c|c|c|}
			\hline
			Распределение   &      $Q_1^T$	& $Q_3^T$ & $X_1^T$ & $X_2^T$ & $P_B^T$	\\
			\hline
			Нормальное & -0.674& 0.674 & -2.698 	&  2.698 	& 0.007 \\
			\hline
			Коши & -1	& 1		&  -4		& 4			& 0.156 \\
			\hline
			Лапласа	&-0.490	& 0.490	& -1.961	& 1.961		& 0.063\\
			\hline
			Пуассона & 8		& 12	& 2			& 18		& 0.008 \\
			\hline
			Равномерное &-0.866 & 0.866	& -3.464 	& 3.464 	& 0	\\
			
			\hline
		\end{tabular}
\caption{Доля выбросов}\label{outliers}
\end{table}
\section{Обсуждение}
\subsection{Доля и теоретическая вероятность выбросов}
По данным, приведенных в таблицах, можно сделать вывод, что увеличение выборки ведет к приближению доли выбросов к теоретической оценке. Доля выбросов для распределения Коши значительно больше, чем для остальных распределений. В равномерном распределении выбросы отсутствуют.

Боксплоты Тьюки веьма наглядно визуализируют характеристики выборок, проводить анализ по ним намного проще, чем по табличным данным.
\section*{Примечание}
С исходным кодом работы и данного отчета можно ознакомиться в репозитории\;\url{https://github.com/Stasychbr/MatStat}
\end{document}
