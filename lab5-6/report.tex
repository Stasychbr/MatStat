%%%%%%%%%%%%%%%%%%%%%%%%%%%%%%%%%%%%%%%%%
% Homework Assignment Article
% LaTeX Template
% Version 1.3.5r (2018-02-16)
%
% This template has been downloaded from:
% /cl.uni-heidelberg.de/~zimmermann/
%
% Original author:
% Victor Zimmermann (zimmermann@cl.uni-heidelberg.de)
%
% License:
% CC BY-SA 4.0 (https://creativecommons.org/licenses/by-sa/4.0/)
%
%%%%%%%%%%%%%%%%%%%%%%%%%%%%%%%%%%%%%%%%%

%----------------------------------------------------------------------------------------

\documentclass[a4paper,12pt]{article} % Uses article class in A4 format

%----------------------------------------------------------------------------------------
%	FORMATTING
%----------------------------------------------------------------------------------------

\setlength{\parskip}{0pt}
\setlength{\parindent}{0pt}
\setlength{\voffset}{-15pt}

%----------------------------------------------------------------------------------------
%	PACKAGES AND OTHER DOCUMENT CONFIGURATIONS
%----------------------------------------------------------------------------------------

\usepackage[a4paper, margin=2.5cm]{geometry} % Sets margin to 2.5cm for A4 Paper
\usepackage[onehalfspacing]{setspace} % Sets Spacing to 1.5
%\linespread{2}

\usepackage[T2A]{fontenc} % Use European encoding
\usepackage[utf8]{inputenc} % Use UTF-8 encoding
%\usepackage{lmodern}
%\usepackage{charter} % Use the Charter font
%\usepackage{microtype} % Slightly tweak font spacing for aesthetics

\usepackage[english, russian]{babel} % Language hyphenation and typographical rules

\usepackage{amsthm, amsmath, amssymb} % Mathematical typesetting
\usepackage{marvosym, wasysym} % More symbols
\usepackage{float} % Improved interface for floating objects
\usepackage[final, colorlinks = true, 
            linkcolor = black, 
            citecolor = black,
            urlcolor = black]{hyperref} % For hyperlinks in the PDF
\usepackage{graphicx, multicol} % Enhanced support for graphics
\usepackage{xcolor} % Driver-independent color extensions
\usepackage{rotating} % Rotation tools
\usepackage{listings, style/lstlisting} % Environment for non-formatted code, !uses style file!
\usepackage{pseudocode} % Environment for specifying algorithms in a natural way
\usepackage{style/avm} % Environment for f-structures, !uses style file!
\usepackage{booktabs} % Enhances quality of tables

\usepackage{tikz-qtree} % Easy tree drawing tool
\tikzset{every tree node/.style={align=center,anchor=north},
         level distance=2cm} % Configuration for q-trees
\usepackage{style/btree} % Configuration for b-trees and b+-trees, !uses style file!

\usepackage{titlesec} % Allows customization of titles
%\renewcommand\thesection{\arabic{section}.} % Arabic numerals for the sections
\titleformat{\section}{\large}{\thesection}{1em}{}
%\renewcommand\thesubsection{\alph{subsection})} % Alphabetic numerals for subsections
\titleformat{\subsection}{\large}{\thesubsection}{1em}{}
%\renewcommand\thesubsubsection{\roman{subsubsection}.} % Roman numbering for subsubsections
\titleformat{\subsubsection}{\large}{\thesubsubsection}{1em}{}

\usepackage[all]{nowidow} % Removes widows

\usepackage{csquotes} % Context sensitive quotation facilities

\usepackage[ddmmyyyy]{datetime} % Uses YEAR-MONTH-DAY format for dates
\renewcommand{\dateseparator}{.} % Sets dateseparator to '-'

\usepackage{fancyhdr} % Headers and footers
\pagestyle{fancy} % All pages have headers and footers
\fancyhead{}\renewcommand{\headrulewidth}{0pt} % Blank out the default header
\fancyfoot[L]{\textsc{}} % Custom footer text
\fancyfoot[C]{} % Custom footer text
\fancyfoot[R]{\thepage} % Custom footer text

\newcommand{\note}[1]{\marginpar{\scriptsize \textcolor{red}{#1}}} % Enables comments in red on margin

\usepackage{microtype} % Slightly tweak font spacing for aesthetics
\usepackage{amsthm, amssymb, amsmath, amsfonts, nccmath}
\usepackage{nicefrac}
\usepackage{float} % Improved interface for floating objects
\usepackage{graphicx, multicol} % Enhanced support for graphics
\usepackage{pdfrender,xcolor}
%\usepackage{breqn}
\usepackage{mathtools}
\usepackage{tikz}
\usepackage{marvosym, wasysym} % More symbols
\usepackage{rotating} % Rotation tools
\usepackage{censor}

\usepackage{algorithm}
\usepackage{algpseudocode}

\DeclareMathOperator{\cov}{cov}
\DeclareMathOperator{\med}{med}
\DeclareMathOperator{\sign}{sign}
%----------------------------------------------------------------------------------------

\begin{document}

%----------------------------------------------------------------------------------------
%	TITLE SECTION
%----------------------------------------------------------------------------------------

\large
\begin{center}
    Санкт-Петербургский политехнический университет\\
    Высшая школа прикладной математики и\\вычислительной физики, ИПММ\\
    \vspace{5em}
    Направление подготовки\\
    01.03.02 «Прикладная математика и информатика»\\
    \vspace{3em}
    Отчет по лабораторным работам №5-6\\
    по дисциплине «Математическая статистика»
    \vspace{15em}
\end{center}
Выполнил студент гр. 3630102/80201\\
Кирпиченко С. Р.\\
Руководитель\\
Баженов А. Н.
\vspace{7em}
\begin{center}
    Санкт-Петербург\\
    2021
\end{center}
\thispagestyle{empty}
\newpage
\tableofcontents
\addtocontents{toc}{~\hfill\textbf{Страница}\par}
\newpage
\listoffigures
\addtocontents{lof}{~\hfill\textbf{Страница}\par}
\newpage
\listoftables
\addtocontents{lot}{~\hfill\textbf{Страница}\par}
\thispagestyle{empty}
\newpage
%----------------------------------------------------------------------------------------
%	TITLE SECTION
%----------------------------------------------------------------------------------------
\section{Постановка задачи}
\begin{enumerate}
    \item Сгенерировать двумерные выборки размерами $20,\,60,\,100$ для нормального двумерного распределения $N(x,y,0,0,1,1,\rho)$.\\
    Коэффициент корреляции $\rho$ взять равным $0,\,0.5,\,0.9$.\\
    Каждая выборка генерируется $1000$ раз и для неё вычисляются: среднее значение, среднее значение квадрата и дисперсия коэффициентов корреляции Пирсона, Спирмена и квадрантного коэффициента корреляции.\\
    Повторить все вычисления для смеси нормальных распределений:
    \begin{equation*}
        f(x,y)=0.9N(x,y,0,0,1,1,0.9)+0.1N(x,y,0,0,10,10,-0.9).
    \end{equation*}
    Изобразить сгенерированные точки на плоскости и нарисовать эллипс
    равновероятности.
\end{enumerate}
\section{Теория}
\subsection{Двумерное нормальное распределение}
Двумерная случайная величина $(X, Y)$ называется распределенной нормально, если её плотность вероятности определяется формулой
\begin{align}
    &N(x,y,\overline{x},\overline{y},\sigma_x,\sigma_y,\rho_{XY}^{})=\frac{1}{2\pi\sigma_x\sigma_y\sqrt{1-\rho_{XY}^2}}\times\nonumber\\
    &\times\exp\left\{-\frac{1}{2(1-\rho_{XY}^2)}\left[\frac{\left(x-\overline{x}\right)^2}{\sigma_x^2}-2\rho_{XY}^{}\frac{(x-\overline{x})(y-\overline{y})}{\sigma_x\sigma_y}+\frac{\left(y-\overline{y}\right)^2}{\sigma_y^2}\right]\right\},
\end{align}
где $\overline{x},\,\overline{y},\sigma_x,\sigma_y$ - математические ожидания и средние квадратические отклонения компонент $X,\,Y$ соответственно, а $\rho_{XY}^{}\:-$ коэффициент корреляции. 
\subsection{Корреляционный момент и коэффициент корреляции}
\textit{Корреляционный момент} (\textit{ковариация}) двух случайных величин $X, Y$:
\begin{equation}
    K_{X Y} = \cov{(X,Y)}=\mathbf{M}\left[(X-\overline{x})(Y-\overline{y})\right].
\end{equation}
\textit{Коэффициент корреляции} $\rho_{X Y}$ случайных величин $X,Y$:
\begin{equation}
    \rho_{X Y}^{}=\frac{K_{X Y}}{\sigma_x\sigma_y}.
\end{equation}
\textit{Ковариационной матрицей} случайного вектора $(X,Y)$ называется симметричная матрица вида
\begin{equation}
    K=\begin{pmatrix}
    D_X & K_{X Y} \\
    K_{Y X} & D_Y
    \end{pmatrix}.
\end{equation}
\textit{Кореляционной матрицей} случайного вектора $(X,Y)$ называется нормированная ковариационная матрица вида
\begin{equation}
    R=\begin{pmatrix}
    1 & \rho_{X Y}^{} \\
    \rho_{Y X}^{} & 1
    \end{pmatrix}.
\end{equation}
\subsection{Выборочные коэффициенты корреляции}
\subsubsection{Выборочный коэффициент корреляции Пирсона}
\textit{Выборочный коэффициент корреляции Пирсона}:
\begin{equation}
    r=\frac{\frac{1}{n}\sum_{i=1}^n \left(x_i-\overline{x}\right)\left(y_i-\overline{y}\right)}{\sqrt{\frac{1}{n}\sum_{i=1}^n\left(x_i-\overline{x}\right)^2 \frac{1}{n}\sum_{i=1}^n\left(y_i-\overline{y}\right)^2}}=\frac{K_{XY}}{s_X s_Y},
\end{equation}
где $K,\,s_X^2,\,s_Y^2\:-$ выборочные ковариация и дисперсии случайных величин $X, Y$.
\subsubsection{Выборочный квадрантный коэффициент корреляции}
\begin{equation}
    r_Q=\frac{(n_1+n_3)-(n_2+n_4)}{n},
\end{equation}
где $n_1,n_2,n_3,n_4\:-$ количества точек с координатами $(x_i,y_i)$, попавшими соответственно в I, II, III и IV квадранты декартовой системы с осями $x'=x-\med{x},\,y'=y-\med{y}$ и с центром в точке с координатами $(\med{x},\med{y})$.
\subsubsection{Выборочный коэффициент ранговой корреляции Спирмена}
Обозначим ранги, соотвествующие значениям переменной $X$, через $u$, а ранги, соответствующие значениям переменной $Y$, $-$ через $v$.
\\\\
\textit{Выборочный коэффициент ранговой корреляции Спирмена}:
\begin{equation}
    r_S=\frac{\frac{1}{n}\sum_{i=1}^n \left(u_i-\overline{u}\right)\left(v_i-\overline{v}\right)}{\sqrt{\frac{1}{n}\sum_{i=1}^n\left(u_i-\overline{u}\right)^2 \frac{1}{n}\sum_{i=1}^n\left(v_i-\overline{v}\right)^2}},
\end{equation}
где $\overline{u}=\overline{v}=\frac{1+2+...+n}{n}=\frac{n+1}{2}\,-$ среднее значение рангов.
\subsection{Эллипсы рассеивания}
Уравнение проекции эллипса рассеивания на плоскость $xOy$:
\begin{equation}\label{eq:ellipse}
    \frac{\left(x-\overline{x}\right)^2}{\sigma_x^2}-2\rho_{XY}^{}\frac{(x-\overline{x})(y-\overline{y})}{\sigma_x\sigma_y}+\frac{\left(y-\overline{y}\right)^2}{\sigma_y^2}=C,\;\;C\,-\,\text{const}.
\end{equation}
Центр эллипса \eqref{eq:ellipse} находится в точке с координатами $(\overline{x},\overline{y})$, оси симметрии эллипса составляют с осью $Ox$ углы, определяемые уравнением
\begin{equation}
    \tan{2\alpha}=\frac{2\rho_{XY}^{}\sigma_x\sigma_y}{\sigma_x^2-\sigma_y^2}.
\end{equation}
\subsection{Простая линейная регрессия}
\subsubsection{Модель простой линейной регрессии}
Регрессионую модель описания данных называют \textit{простой линейной регрессией}, если
\begin{equation}
    y_i=\beta_0 + \beta_1 x_i + \varepsilon_i,\;\;i=1,...,n,
\end{equation}
где $x_1, ..., x_n\:-$ заданные числа (значения фактора); $y_1,...,y_n\:-$ наблюдаемые значения отклика; $\varepsilon_1,...,\varepsilon_n\:-$ независимые, нормально распределенные $N(0,\sigma)$ с нулевым математическим ожиданием и одинаковой (неизвестной) дисперсией случайные величины (ненаблюдаемые); $\beta_0,\:\beta_1\:-$ неизвестные параметры, подлежащие оцениванию.
\subsubsection{Метод наименьших квадратов}
\textit{Метод наименьших квадратов} (МНК):
\begin{equation}
    Q\left(\beta_0,\beta_1\right)=\sum_{i=1}^n \varepsilon_i^2= \sum_{i=1}^n\left(y_i-\beta_0-\beta_1 x_i\right)^2\to\min_{\beta_0,\beta_1}.
\end{equation}
\subsubsection{Расчётные формулы для МНК-оценок}
МНК-оценки параметров $\beta_0$ и $\beta_1$:
\begin{equation}
    \widehat{\beta}_1=\frac{\overline{xy}-\overline{x}\cdot\overline{y}}{\overline{x^2}-(\overline{x})^2},
\end{equation}
\begin{equation}
    \widehat{\beta}_0=\overline{y}-\overline{x}\widehat{\beta}_1.
\end{equation}
\subsection{Робастные оценки коэффициентов линейной регрессии}
\textit{Метод наименьших модулей}:
\begin{equation}
    \sum_{i=1}^n |y_i-\beta_0-\beta_1 x_i|\to \min_{\beta_0,\beta_1}.
\end{equation}
\begin{equation}
    \widehat{\beta}_{1R}=r_Q\frac{q_y^*}{q_x^*},
\end{equation}
\begin{equation}
    \widehat{\beta}_{0R}=\med{y}-\widehat{\beta}_{1R}\med{x},
\end{equation}
\begin{equation}
    r_Q=\frac{1}{n}\sum_{i=1}^n \sign{(x_i-\med{x})}\sign{(y_i-\med{y})},
\end{equation}
\begin{equation}
    q_y^*=\frac{y_{(j)}-y_{(l)}}{k_q(n)},\;\;q_x^*=\frac{x_{(j)}-x_{(l)}}{k_q(n)}
\end{equation}
\begin{equation*}
    l=\begin{cases}
        \displaystyle\;\;[n/4]+1&\text{при}\;\;n/4\;\;\text{дробном,}\\
        \displaystyle\;\;\;\;\;\;\;n/4&\text{при}\;\;n/4\;\;\text{целом}.
    \end{cases}
\end{equation*}
\begin{equation*}
    j=n-l+1.
\end{equation*}
\begin{equation*}
    \sign{z} = \begin{cases}
    \;\:\:1&\text{при}\;\;z>0,\\
    \;\:\:0&\text{при}\;\;z=0,\\
    -1&\text{при}\;\;z<0.
    \end{cases}
\end{equation*}
Уравнение регрессии здесь имеет вид
\begin{equation}
    y = \widehat{\beta}_{0R}+\widehat{\beta}_{1R}\cdot x.
\end{equation}
\begin{equation*}
    k_q(20)=1.491.
\end{equation*}
\subsection{Метод максимального правдоподобия}
$L(x_1,...,x_n,\theta)\;-$ функция правдоподобия(ФП), рассматриваемая как функция неизвестного параметра $\theta$:
\begin{equation}
    L(x_1,...,x_n,\theta)=f(x_1,\theta)f(x_2,\theta)...f(x_n,\theta).
\end{equation}

\section{Реализация}
Лабораторная работа выполнена на языке Python 3.9 с использованием библиотек numpy, scipy, matplotlib, seaborn.
\section{Результаты}
\subsection{Выборочные коэффициенты корреляции}
\begin{table}[H]
    \centering
    \begin{tabular}{|c|c|c|c|}
\hline
\multicolumn{4}{|c|}{$\rho=0$}\\
\hline
&$r$&$r_S$&$r_Q$\\
\hline
E($z$)&0.0026&0.003&0.0029\\
\hline
E($z^2$)&0.0537&0.055&0.0555\\
\hline
D($z$)&0.0537&0.055&0.0554\\
\hline
\multicolumn{4}{|c|}{$\rho=0.5$}\\
\hline
&$r$&$r_S$&$r_Q$\\
\hline
E($z$)&0.4865&0.46&0.338\\
\hline
E($z^2$)&0.2679&0.246&0.1576\\
\hline
D($z$)&0.0313&0.034&0.0433\\
\hline
\multicolumn{4}{|c|}{$\rho=0.9$}\\
\hline
&$r$&$r_S$&$r_Q$\\
\hline
E($z$)&0.8948&0.866&0.7068\\
\hline
E($z^2$)&0.803&0.755&0.5259\\
\hline
D($z$)&0.0024&0.004&0.0263\\
\hline
\end{tabular}
    \caption{Двумерное нормальное распределение, n = 20}
    \label{tab:norm_n_20}
\end{table}
\begin{table}[H]
    \centering
    \begin{tabular}{|c|c|c|c|}
\hline
\multicolumn{4}{|c|}{$\rho=0$}\\
\hline
&$r$&$r_S$&$r_Q$\\
\hline
E($z$)&0.0029&0.002&0.0018\\
\hline
E($z^2$)&0.0167&0.017&0.0172\\
\hline
D($z$)&0.0167&0.017&0.0172\\
\hline
\multicolumn{4}{|c|}{$\rho=0.5$}\\
\hline
&$r$&$r_S$&$r_Q$\\
\hline
E($z$)&0.5004&0.479&0.3371\\
\hline
E($z^2$)&0.2601&0.24&0.1277\\
\hline
D($z$)&0.0097&0.01&0.0141\\
\hline
\multicolumn{4}{|c|}{$\rho=0.9$}\\
\hline
&$r$&$r_S$&$r_Q$\\
\hline
E($z$)&0.8974&0.882&0.7121\\
\hline
E($z^2$)&0.8059&0.779&0.5151\\
\hline
D($z$)&0.0006&0.001&0.008\\
\hline
\end{tabular}
    \caption{Двумерное нормальное распределение, n = 60}
    \label{tab:norm_n_60}
\end{table}
\begin{table}[H]
    \centering
\begin{tabular}{|c|c|c|c|}
\hline
\multicolumn{4}{|c|}{$\rho=0$}\\
\hline
&$r$&$r_S$&$r_Q$\\
\hline
E($z$)&0.002&0.002&-0.0015\\
\hline
E($z^2$)&0.0098&0.01&0.0106\\
\hline
D($z$)&0.0098&0.01&0.0106\\
\hline
\multicolumn{4}{|c|}{$\rho=0.5$}\\
\hline
&$r$&$r_S$&$r_Q$\\
\hline
E($z$)&0.4989&0.478&0.3317\\
\hline
E($z^2$)&0.2547&0.235&0.1188\\
\hline
D($z$)&0.0059&0.006&0.0088\\
\hline
\multicolumn{4}{|c|}{$\rho=0.9$}\\
\hline
&$r$&$r_S$&$r_Q$\\
\hline
E($z$)&0.8989&0.886&0.7107\\
\hline
E($z^2$)&0.8084&0.785&0.5098\\
\hline
D($z$)&0.0004&0.001&0.0048\\
\hline
\end{tabular}
    \caption{Двумерное нормальное распределение, n = 100}
    \label{tab:norm_n_100}
\end{table}
\begin{table}[H]
    \centering
\begin{tabular}{|c|c|c|c|}
\hline
\multicolumn{4}{|c|}{$n=20$}\\
\hline
&$r$&$r_S$&$r_Q$\\
\hline
E($z$)&-0.1029&-0.095&-0.0678\\
\hline
E($z^2$)&0.0588&0.057&0.0526\\
\hline
D($z$)&0.0482&0.048&0.048\\
\hline
\multicolumn{4}{|c|}{$n=60$}\\
\hline
&$r$&$r_S$&$r_Q$\\
\hline
E($z$)&-0.087&-0.083&-0.0567\\
\hline
E($z^2$)&0.0234&0.023&0.0204\\
\hline
D($z$)&0.0158&0.016&0.0172\\
\hline
\multicolumn{4}{|c|}{$n=100$}\\
\hline
&$r$&$r_S$&$r_Q$\\
\hline
E($z$)&-0.0935&-0.089&-0.0589\\
\hline
E($z^2$)&0.0192&0.018&0.0134\\
\hline
D($z$)&0.0104&0.01&0.0099\\
\hline
\end{tabular}
    \caption{Смесь нормальных распределений}
    \label{tab:norm_mix}
\end{table}
\subsection{Эллипсы рассеивания}
\begin{figure}[H]
    \centering
    \includegraphics[width = 15 cm]{plot20.pdf}
    \caption{Двумерное нормальное распределение, n = 20}
    \label{fig:n20}
\end{figure}
\begin{figure}[H]
    \centering
    \includegraphics[width = 15 cm]{plot60.pdf}
    \caption{Двумерное нормальное распределение, n = 60}
    \label{fig:n60}
\end{figure}
\begin{figure}[H]
    \centering
    \includegraphics[width = 15 cm]{plot100.pdf}
    \caption{Двумерное нормальное распределение, n = 100}
    \label{fig:n100}
\end{figure}
\subsection{Оценки коэффициентов линейной регрессии}
\subsubsection{Выборка без возмущений}
Коэффициенты прямых:
\begin{enumerate}
    \item Метод наименьших квадратов: $\hat{\beta_1}=2.1838,\;\hat{\beta_0}=2.3362$;
    \item Метод наименьших модулей: $\hat{\beta_1}=2.0006,\;\hat{\beta_0}=2.4235$.
\end{enumerate}
\begin{figure}[H]
    \centering
    \includegraphics[width = 15 cm]{straight.pdf}
    \caption{Выборка без возмущений}
    \label{fig:straight}
\end{figure}
\subsubsection{Выборка с возмущениями}
Коэффициенты прямых:
\begin{enumerate}
    \item Метод наименьших квадратов: $\hat{\beta_1}=0.5469,\;\hat{\beta_0}=1.8807$;
    \item Метод наименьших модулей: $\hat{\beta_1}=0.8314,\;\hat{\beta_0}=1.8622$.
\end{enumerate}
\begin{figure}[H]
    \centering
    \includegraphics[width = 15 cm]{perm.pdf}
    \caption{Выборка с возмущениями}
    \label{fig:perm}
\end{figure}
\section{Обсуждение}
\subsection{Выборочные коэффициенты корреляции и эллипсы рассеивания}
Для дисперсий выборочных коэффициентов корреляции можно сделать следующие выводы:
\begin{enumerate}
    \item Для двумерного нормального распределения справедлив следующий порядок: $D(r)\leq D(r_S)\leq D(r_Q)$. Коэффициент Пирсона является оптимальным для анализа подобных выборок. 
    \item Для смеси нормальных распределений дисперсии всех трех коэффициентов примерно равны.
\end{enumerate}
Процент попадания элементов выборки в эллипс рассеивания примерно равен теоретическому значению (95\%).
\subsection{Оценки коэффициентов линейной регрессии}
Для выборки без возмущений методы наименьших квадратов и модулей дают схожие хорошие результаты, однако МНМ дает более параллельную к исходной прямую.

Для выборки с возмущениями МНК и МНМ также дают схожие прямые, ввиду рода возмущений коэффициент наклона сильно отличается от эталона, однако метод наименьших модулей показал большую устойчивость.
\newpage
\section*{Примечание}
\begin{thebibliography}{9}
\bibitem{book1} 
 Максимов Ю.Д. Математика. Теория и практика по математической статистике. Конспект-справочник по теории вероятностей : учеб. пособие /
Ю.Д. Максимов; под ред. В.И. Антонова. $-$ СПб. : Изд-во Политехн.
ун-та, 2009. $-$ 395 с. (Математика в политехническом университете).
\end{thebibliography}
\end{document}
