%%%%%%%%%%%%%%%%%%%%%%%%%%%%%%%%%%%%%%%%%
% Homework Assignment Article
% LaTeX Template
% Version 1.3.5r (2018-02-16)
%
% This template has been downloaded from:
% /cl.uni-heidelberg.de/~zimmermann/
%
% Original author:
% Victor Zimmermann (zimmermann@cl.uni-heidelberg.de)
%
% License:
% CC BY-SA 4.0 (https://creativecommons.org/licenses/by-sa/4.0/)
%
%%%%%%%%%%%%%%%%%%%%%%%%%%%%%%%%%%%%%%%%%

%----------------------------------------------------------------------------------------

\documentclass[a4paper,12pt]{article} % Uses article class in A4 format

%----------------------------------------------------------------------------------------
%	FORMATTING
%----------------------------------------------------------------------------------------

\setlength{\parskip}{0pt}
\setlength{\parindent}{0pt}
\setlength{\voffset}{-15pt}

%----------------------------------------------------------------------------------------
%	PACKAGES AND OTHER DOCUMENT CONFIGURATIONS
%----------------------------------------------------------------------------------------

\usepackage[a4paper, margin=2.5cm]{geometry} % Sets margin to 2.5cm for A4 Paper
\usepackage[onehalfspacing]{setspace} % Sets Spacing to 1.5
%\linespread{2}

\usepackage[T2A]{fontenc} % Use European encoding
\usepackage[utf8]{inputenc} % Use UTF-8 encoding
%\usepackage{lmodern}
%\usepackage{charter} % Use the Charter font
%\usepackage{microtype} % Slightly tweak font spacing for aesthetics

\usepackage[english, russian]{babel} % Language hyphenation and typographical rules

\usepackage{amsthm, amsmath, amssymb} % Mathematical typesetting
\usepackage{marvosym, wasysym} % More symbols
\usepackage{float} % Improved interface for floating objects
\usepackage[final, colorlinks = true, 
            linkcolor = black, 
            citecolor = black,
            urlcolor = black]{hyperref} % For hyperlinks in the PDF
\usepackage{graphicx, multicol} % Enhanced support for graphics
\usepackage{xcolor} % Driver-independent color extensions
\usepackage{rotating} % Rotation tools
\usepackage{listings, style/lstlisting} % Environment for non-formatted code, !uses style file!
\usepackage{pseudocode} % Environment for specifying algorithms in a natural way
\usepackage{style/avm} % Environment for f-structures, !uses style file!
\usepackage{booktabs} % Enhances quality of tables

\usepackage{tikz-qtree} % Easy tree drawing tool
\tikzset{every tree node/.style={align=center,anchor=north},
         level distance=2cm} % Configuration for q-trees
\usepackage{style/btree} % Configuration for b-trees and b+-trees, !uses style file!

\usepackage{titlesec} % Allows customization of titles
%\renewcommand\thesection{\arabic{section}.} % Arabic numerals for the sections
\titleformat{\section}{\large}{\thesection}{1em}{}
%\renewcommand\thesubsection{\alph{subsection})} % Alphabetic numerals for subsections
\titleformat{\subsection}{\large}{\thesubsection}{1em}{}
%\renewcommand\thesubsubsection{\roman{subsubsection}.} % Roman numbering for subsubsections
\titleformat{\subsubsection}{\large}{\thesubsubsection}{1em}{}

\usepackage[all]{nowidow} % Removes widows

\usepackage{csquotes} % Context sensitive quotation facilities

\usepackage[ddmmyyyy]{datetime} % Uses YEAR-MONTH-DAY format for dates
\renewcommand{\dateseparator}{.} % Sets dateseparator to '-'

\usepackage{fancyhdr} % Headers and footers
\pagestyle{fancy} % All pages have headers and footers
\fancyhead{}\renewcommand{\headrulewidth}{0pt} % Blank out the default header
\fancyfoot[L]{\textsc{}} % Custom footer text
\fancyfoot[C]{} % Custom footer text
\fancyfoot[R]{\thepage} % Custom footer text

\newcommand{\note}[1]{\marginpar{\scriptsize \textcolor{red}{#1}}} % Enables comments in red on margin

\usepackage{microtype} % Slightly tweak font spacing for aesthetics
\usepackage{amsthm, amssymb, amsmath, amsfonts, nccmath}
\usepackage{nicefrac}
\usepackage{float} % Improved interface for floating objects
\usepackage{graphicx, multicol} % Enhanced support for graphics
\usepackage{pdfrender,xcolor}
%\usepackage{breqn}
\usepackage{mathtools}
\usepackage{tikz}
\usepackage{marvosym, wasysym} % More symbols
\usepackage{rotating} % Rotation tools
\usepackage{censor}

\usepackage{algorithm}
\usepackage{algpseudocode}

\DeclareMathOperator{\cov}{cov}
\DeclareMathOperator{\med}{med}
\DeclareMathOperator{\sign}{sign}
%----------------------------------------------------------------------------------------

\begin{document}

%----------------------------------------------------------------------------------------
%	TITLE SECTION
%----------------------------------------------------------------------------------------

\large
\begin{center}
    Санкт-Петербургский политехнический университет\\
    Высшая школа прикладной математики и\\вычислительной физики, ИПММ\\
    \vspace{5em}
    Направление подготовки\\
    01.03.02 «Прикладная математика и информатика»\\
    \vspace{3em}
    Отчет по лабораторной работе №7\\
    по дисциплине «Математическая статистика»
    \vspace{15em}
\end{center}
Выполнил студент гр. 3630102/80201\\
Кирпиченко С. Р.\\
Руководитель\\
Баженов А. Н.
\vspace{7em}
\begin{center}
    Санкт-Петербург\\
    2021
\end{center}
\thispagestyle{empty}
\newpage
\tableofcontents
\addtocontents{toc}{~\hfill\textbf{Страница}\par}
\newpage
\listoftables
\addtocontents{lot}{~\hfill\textbf{Страница}\par}
\thispagestyle{empty}
\newpage
%----------------------------------------------------------------------------------------
%	TITLE SECTION
%----------------------------------------------------------------------------------------
\section{Постановка задачи}
Сгенерировать выборку объёмом 100 элементов для нормального распределения $N(x,0,1)$. По сгенерированной выборке оценить параметры $\mu$ и $\sigma$ нормального закона методом максимального правдоподобия.
В качестве основной гипотезы $H_0$ будем считать, что сгенерированное
распределение имеет вид $N(x,\hat{\mu},\hat{\sigma})$. Проверить основную гипотезу, используя критерий согласия $\chi^2$. В качестве уровня значимости взять $\alpha=0.05$. Привести таблицу вычислений $\chi^2$.

Исследовать точность (чувствительность) критерия $\chi^2$ - сгенерировать выборки равномерного распределения и распределения Лапласа малого объема (20 элементов). Проверить их на нормальность.
\section{Теория}
\subsection{Метод максимального правдоподобия}
$L(x_1,...,x_n,\theta)\;-$ функция правдоподобия(ФП), рассматриваемая как функция неизвестного параметра $\theta$:
\begin{equation}
    L(x_1,...,x_n,\theta)=f(x_1,\theta)f(x_2,\theta)...f(x_n,\theta).
\end{equation}
\textit{Оценка максимального правдоподобия}:
\begin{equation}
    \widehat{\theta}_{\text{мп}}=\arg{\max_\theta{L(x_1,...,x_n,\theta)}}.
\end{equation}
Система уравнений правдоподобия (в случае дифференцируемости функции правдоподобия):
\begin{equation}
    \frac{\partial L}{\partial\theta_k}=0\;\;\text{или}\;\;\frac{\partial\ln{L}}{\partial\theta_k}=0,\;\;k=1,...,m.
\end{equation}
\subsection{Проверка гипотезы о законе распределения генеральной совокупности. Метод хи-квадрат}
Выдвинута гипотеза $H_0$ о генеральном законе распределения с функцией
распределения $F(x)$.\\\\
Рассматриваем случай, когда гипотетическая функция распределения $F(x)$ не содержит неизвестных параметров.
\subsubsection*{Правило проверки гипотезы о законе распределения по методу $\chi^2$}
\begin{enumerate}
    \item Выбираем уровень значимости $\alpha$.
    \item По таблице \cite[с. 358]{book1} находим квантиль $\chi_{1-\alpha}^2(k-1)$ распределения хи-квадрат с $k-1$ степенями свободы порядка $1-\alpha$.
    \item Вычисляем вероятности $p_i=P(X\in\Delta_i), i = 1,...,k$, с помощью гипотетической функции распределения $F(x)$.
    \item Находим частоты $n_i$ попадания элементов выборки в подмножества $\Delta_i,i=1,...,k$.
    \item Вычисляем выборочное значение статистики критерия $\chi^2$:
        \begin{equation*}
            \chi^2_{\text{В}}=\sum_{i=1}^k\frac{(n_i-np_i)^2}{np_i}.
        \end{equation*}
    \item Сравниваем $\chi^2_{\text{В}}$ и квантиль $\chi_{1-\alpha}^2(k-1)$.
        \begin{enumerate}
            \item Если $\chi^2_{\text{В}}<\chi_{1-\alpha}^2(k-1)$, то гипотеза $H_0$ на данном этапе проверки принимается.
            \item Если $\chi^2_{\text{В}}\geq\chi_{1-\alpha}^2(k-1)$, то гипотеза $H_0$ отвергается, выбирается одно из альтернативных распределений, и процедура проверки повторяется.

        \end{enumerate}
\end{enumerate}

\section{Реализация}
Лабораторная работа выполнена на языке Python 3.9 с использованием библиотек numpy, scipy, matplotlib, seaborn.
\section{Результаты}
\subsection{Проверка гипотезы о законе распределения генеральной совокупности. Метод хи-квадрат}
Метод максимального правдоподобия:
$$\hat{\mu}\approx0.08\quad\hat{\sigma}\approx1.08$$
Критерий согласия $\chi^2$:
\begin{enumerate}
    \item Количество промежутков $k=1.72\sqrt[3]{n}=8$
    \item Уровень значимости $\alpha=0.05$
    \item $\chi^2_{1-\alpha}(k-1)=\chi^2_{0.95}(7)=14.07$
\end{enumerate}
\begin{table}[H]
    \centering
    \begin{tabular}{|c|c|c|c|c|c|c|}
    \hline
    i&$\Delta_i$&$n_i$&$p_i$&$np_i$&$n_i-np_i$&$\dfrac{(n_i-np_i)^2}{np_i}$\\
    \hline
    1&$(-\infty,\;-1.1)$&11&0.138&13.8&-2.8&0.5674\\
    \hline
    2&$(-1.1,\;-0.73)$&9&0.0886&8.86&0.14&0.0022\\
    \hline
    3&$(-0.73,\;-0.37)$&14&0.114&11.4&2.6&0.5928\\
    \hline
    4&$(-0.37,\;0.0)$&16&0.1309&13.09&2.91&0.6478\\
    \hline
    5&$(0.0,\;0.37)$&9&0.1341&13.41&-4.41&1.4486\\
    \hline
    6&$(0.37,\;0.73)$&11&0.1225&12.25&-1.25&0.1282\\
    \hline
    7&$(0.73,\;1.1)$&15&0.0999&9.99&5.01&2.5094\\
    \hline
    8&$(1.1,\;\infty)$&15&0.172&17.2&-2.2&0.2815\\
    \hline
    $\Sigma$&--&100&1&100&0&6.18=$\chi^2_B$\\
    \hline
    \end{tabular}
    \caption{Проверка гипотезы $H_0$ на нормальной выборке}
    \label{tab:tab1}
\end{table}
$\chi^2_B<\chi^2_{0.95}(7)\Rightarrow$ на текущем этапе гипотеза $H_0$ о том, что генеральная выборка имеет распределение $N(x,\hat{\mu},\hat{\sigma})$, принимается.
\subsection{Исследование на чувствительность}
Рассмотрим выборку, сгенерированную по распределению Лапласа.
$$\hat{\mu}\approx-0.09\quad\hat{\sigma}\approx0.89$$
Критерий согласия $\chi^2$:
\begin{enumerate}
    \item Количество промежутков $k=1.72\sqrt[3]{n}=5$
    \item Уровень значимости $\alpha=0.05$
    \item $\chi^2_{1-\alpha}(k-1)=\chi^2_{0.95}(4)=9.49$
\end{enumerate}
\begin{table}[H]
    \centering
    \begin{tabular}{|c|c|c|c|c|c|c|}
    \hline
    i&$\Delta_i$&$n_i$&$p_i$&$np_i$&$n_i-np_i$&$\dfrac{(n_i-np_i)^2}{np_i}$\\
    \hline
    1&$(\infty,\;-1.1)$&4&0.1292&2.58&1.42&0.7753\\
    \hline
    2&$(-1.1,\;-0.37)$&2&0.2495&4.99&-2.99&1.7916\\
    \hline
    3&$(-0.37,\;0.37)$&8&0.3171&6.34&1.66&0.4336\\
    \hline
    4&$(0.37,\;1.1)$&4&0.213&4.26&-0.26&0.0159\\
    \hline
    5&$(1.1,\;\infty)$&2&0.0912&1.82&0.18&0.0171\\
    \hline
    $\Sigma$&--&20&1.0&20.0&0&3.03$=\chi^2_B$\\
    \hline
    \end{tabular}
    \caption{Проверка гипотезы $H_0$ на выборке, сгенерированной по распределению Лапласа}
    \label{tab:tab2}
\end{table}
Рассмотрим выборку, сгенерированную по равномерному распределению.
$$\hat{\mu}\approx-0.09\quad\hat{\sigma}\approx0.88$$
\begin{table}[H]
    \centering
    \begin{tabular}{|c|c|c|c|c|c|c|}
    \hline
    i&$\Delta_i$&$n_i$&$p_i$&$np_i$&$n_i-np_i$&$\dfrac{(n_i-np_i)^2}{np_i}$\\
    \hline
    1&$(\infty,\;-1.1)$&3&0.127&2.54&0.46&0.0836\\
    \hline
    2&$(-1.1,\;-0.37)$&6&0.2516&5.03&0.97&0.1863\\
    \hline
    3&$(-0.37,\;0.37)$&4&0.3207&6.41&-2.41&0.9087\\
    \hline
    4&$(0.37,\;1.1)$&5&0.2128&4.26&0.74&0.1297\\
    \hline
    5&$(1.1,\;\infty)$&2&0.0879&1.76&0.24&0.0333\\
    \hline
    $\Sigma$&--&20&1.0&20.0&0&1.34$=\chi^2_B$\\
    \hline
    \end{tabular}
    \caption{Проверка гипотезы $H_0$ на выборке, сгенерированной по равномерному распределению}
    \label{tab:tab3}
\end{table}
Видим, что в обоих случаях гипотеза принимается.
\section{Обсуждение}
Оценки максимального правдоподобия показали свою состоятельность на выборке из 100 элементов, распределенной нормально. Погрешность найденных параметров наблюдается во втором знаке после запятой.

На малых выборках метод хи-квадрат при проверке на нормальность не различает выборки, распределенные равномерно и по закону Лапласа, подтверждая гипотезу в обоих случаях. Это обусловлено теоремой К. Пирсона: статистика критерия $\chi^2$ асимптотически при $n\rightarrow\infty$ распределена по закону $\chi^2$ с $k-1$ степенями свободы. То есть на малых выборках теория ничего не гарантирует.
\section*{Примечание}
С исходным кодом работы и данного отчета можно ознакомиться в репозитории\;\url{https://github.com/Stasychbr/MatStat}
\newpage
\begin{thebibliography}{9}
\bibitem{book1} 
 Максимов Ю.Д. Математика. Теория и практика по математической статистике. Конспект-справочник по теории вероятностей : учеб. пособие /
Ю.Д. Максимов; под ред. В.И. Антонова. $-$ СПб. : Изд-во Политехн.
ун-та, 2009. $-$ 395 с. (Математика в политехническом университете).
\end{thebibliography}
\end{document}
