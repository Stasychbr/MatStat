%%%%%%%%%%%%%%%%%%%%%%%%%%%%%%%%%%%%%%%%%
% Homework Assignment Article
% LaTeX Template
% Version 1.3.5r (2018-02-16)
%
% This template has been downloaded from:
% /cl.uni-heidelberg.de/~zimmermann/
%
% Original author:
% Victor Zimmermann (zimmermann@cl.uni-heidelberg.de)
%
% License:
% CC BY-SA 4.0 (https://creativecommons.org/licenses/by-sa/4.0/)
%
%%%%%%%%%%%%%%%%%%%%%%%%%%%%%%%%%%%%%%%%%

%----------------------------------------------------------------------------------------

\documentclass[a4paper,12pt]{article} % Uses article class in A4 format

%----------------------------------------------------------------------------------------
%	FORMATTING
%----------------------------------------------------------------------------------------

\setlength{\parskip}{0pt}
\setlength{\parindent}{0pt}
\setlength{\voffset}{-15pt}

%----------------------------------------------------------------------------------------
%	PACKAGES AND OTHER DOCUMENT CONFIGURATIONS
%----------------------------------------------------------------------------------------

\usepackage[a4paper, margin=2.5cm]{geometry} % Sets margin to 2.5cm for A4 Paper
\usepackage[onehalfspacing]{setspace} % Sets Spacing to 1.5
%\linespread{2}

\usepackage[T2A]{fontenc} % Use European encoding
\usepackage[utf8]{inputenc} % Use UTF-8 encoding
%\usepackage{lmodern}
%\usepackage{charter} % Use the Charter font
%\usepackage{microtype} % Slightly tweak font spacing for aesthetics

\usepackage[english, russian]{babel} % Language hyphenation and typographical rules

\usepackage{amsthm, amsmath, amssymb} % Mathematical typesetting
\usepackage{marvosym, wasysym} % More symbols
\usepackage{float} % Improved interface for floating objects
\usepackage[final, colorlinks = true, 
            linkcolor = black, 
            citecolor = black,
            urlcolor = black]{hyperref} % For hyperlinks in the PDF
\usepackage{graphicx, multicol} % Enhanced support for graphics
\usepackage{xcolor} % Driver-independent color extensions
\usepackage{rotating} % Rotation tools
\usepackage{listings, style/lstlisting} % Environment for non-formatted code, !uses style file!
\usepackage{pseudocode} % Environment for specifying algorithms in a natural way
\usepackage{style/avm} % Environment for f-structures, !uses style file!
\usepackage{booktabs} % Enhances quality of tables

\usepackage{tikz-qtree} % Easy tree drawing tool
\tikzset{every tree node/.style={align=center,anchor=north},
         level distance=2cm} % Configuration for q-trees
\usepackage{style/btree} % Configuration for b-trees and b+-trees, !uses style file!

\usepackage{titlesec} % Allows customization of titles
%\renewcommand\thesection{\arabic{section}.} % Arabic numerals for the sections
\titleformat{\section}{\large}{\thesection}{1em}{}
%\renewcommand\thesubsection{\alph{subsection})} % Alphabetic numerals for subsections
\titleformat{\subsection}{\large}{\thesubsection}{1em}{}
%\renewcommand\thesubsubsection{\roman{subsubsection}.} % Roman numbering for subsubsections
\titleformat{\subsubsection}{\large}{\thesubsubsection}{1em}{}

\usepackage[all]{nowidow} % Removes widows

\usepackage{csquotes} % Context sensitive quotation facilities

\usepackage[ddmmyyyy]{datetime} % Uses YEAR-MONTH-DAY format for dates
\renewcommand{\dateseparator}{.} % Sets dateseparator to '-'

\usepackage{fancyhdr} % Headers and footers
\pagestyle{fancy} % All pages have headers and footers
\fancyhead{}\renewcommand{\headrulewidth}{0pt} % Blank out the default header
\fancyfoot[L]{\textsc{}} % Custom footer text
\fancyfoot[C]{} % Custom footer text
\fancyfoot[R]{\thepage} % Custom footer text

\newcommand{\note}[1]{\marginpar{\scriptsize \textcolor{red}{#1}}} % Enables comments in red on margin

\usepackage{microtype} % Slightly tweak font spacing for aesthetics
\usepackage{amsthm, amssymb, amsmath, amsfonts, nccmath}
\usepackage{nicefrac}
\usepackage{float} % Improved interface for floating objects
\usepackage{graphicx, multicol} % Enhanced support for graphics
\usepackage{pdfrender,xcolor}
%\usepackage{breqn}
\usepackage{mathtools}
\usepackage{tikz}
\usepackage{marvosym, wasysym} % More symbols
\usepackage{rotating} % Rotation tools
\usepackage{censor}

\usepackage{algorithm}
\usepackage{algpseudocode}
%----------------------------------------------------------------------------------------

\begin{document}

%----------------------------------------------------------------------------------------
%	TITLE SECTION
%----------------------------------------------------------------------------------------

\large
\begin{center}
    Санкт-Петербургский политехнический университет\\
    Высшая школа прикладной математики и\\вычислительной физики, ИПММ\\
    \vspace{5em}
    Направление подготовки\\
    01.03.02 «Прикладная математика и информатика»\\
    \vspace{3em}
    Отчет по лабораторной работе №4\\
    по дисциплине «Математическая статистика»
    \vspace{15em}
\end{center}
Выполнил студент гр. 3630102/80201\\
Кирпиченко С. Р.\\
Руководитель\\
Баженов А. Н.
\vspace{7em}
\begin{center}
    Санкт-Петербург\\
    2021
\end{center}
\thispagestyle{empty}
\newpage
\tableofcontents
\addtocontents{toc}{~\hfill\textbf{Страница}\par}
\newpage
\listoffigures
\addtocontents{lof}{~\hfill\textbf{Страница}\par}
\newpage
\listoftables
\addtocontents{lot}{~\hfill\textbf{Страница}\par}
\thispagestyle{empty}
\newpage
%----------------------------------------------------------------------------------------
%	TITLE SECTION
%----------------------------------------------------------------------------------------
\section{Постановка задачи}
Для 5 распределений:
\begin{itemize}
    \item Нормальное распределение $N(x, 0, 1)$
    \item Распределение Коши $C(x, 0, 1)$
    \item Распределение Лапласа $L(x, 0, \frac{1}{\sqrt{2}})$
    \item Распределение Пуассона $P(k, 10)$
    \item Равномерное распределение $U(x,-\sqrt{3},\sqrt{3})$
\end{itemize}
Сгенерировать выборки размером 20, 60 и 100 элементов. Построить на них эмпирические функции распределения и ядерные оценки плотности распределения на отрезке $[-4;\,4]$ для непрерывных распределений и на отрезке $[6;\,14]$ для распределения Пуассона.
\section{Теория}
\subsection{Эмпирическая функция распределения}
\subsubsection{Статистический ряд}
Статистическим рядом назовем совокупность, состоящую из последовательности $\displaystyle\{z_i\}_{i=1}^k$ попарно различных элементов выборки, расположенных по возрастанию, и последовательности $\displaystyle\{n_i\}_{i=1}^k$ частот, с которыми эти элементы содержатся в выборке.
\subsubsection{Эмпирическая функция распределения}
Эмпирическая функция распределения (э. ф. р.) - относительная частота события $X < x$, полученная по данной выборке:
\begin{equation}
    F_n^*(x)=P^*(X<x).
\end{equation}
\subsubsection{Нахождение э. ф. р.}
\begin{equation}
    F^*(x)=\frac{1}{n}\sum_{z_i<x}n_i.
\end{equation}
$F^*(x)-$ функция распределения дискретной случайной величины $X^*$, заданной таблицей распределения
\begin{table}[H]
    \centering
    \begin{tabular}{|c|c|c|c|c|}
        \hline
         $X^*$&$z_1$&$z_2$&...&$z_k$\\
         \hline
         $P$&$\frac{n_1}{n}$&$\frac{n_2}{n}$&...&$\frac{n_k}{n}$\\
         \hline
    \end{tabular}
    \caption{Таблица распределения}
    \label{tab:my_label}
\end{table}
Эмпирическая функция распределения является оценкой, т. е. приближённым значением, генеральной функции распределения
\begin{equation}
    F_n^*(x)\approx F_X(x).
\end{equation}
\subsection{Оценки плотности вероятности}
\subsubsection{Определение}
Оценкой плотности вероятности $f(x)$ называется функция $\widehat{f}(x)$, построенная на основе выборки, приближённо равная $f(x)$
\begin{equation}
    \widehat{f}(x)\approx f(x).
\end{equation}
\subsubsection{Ядерные оценки}
Представим оценку в виде суммы с числом слагаемых, равным объёму выборки:
\begin{equation}
    \widehat{f}_n(x)=\frac{1}{n h_n}\sum_{i=1}^n K\left(\frac{x-x_i}{h_n}\right).
\end{equation}
$K(u)$ - ядро, т. е. непрерывная функция, являющаяся плотностью вероятности, $x_1,...,x_n$ $-$ элементы выборки, а $\{h_n\}_{n\in\mathbb{N}}$ - последовательность элементов из $\mathbb{R}_+$ такая, что
\begin{equation}
    h_n\xrightarrow[n\to\infty]{}0;\;\;\;n h_n\xrightarrow[n\to\infty]{}\infty.
\end{equation}
Такие оценки называются непрерывными ядерными.\\\\
Гауссово ядро:
\begin{equation}
    K(u)=\frac{1}{\sqrt{2\pi}}e^{-\frac{u^2}{2}}.
\end{equation}
Правило Сильвермана:
\begin{equation}
    h_n=\left(\frac{4\hat{\sigma}^5}{3n}\right)^{1/5}\approx1.06\hat{\sigma}n^{-1/5},
\end{equation}
где $\hat{\sigma}$ - выборочное стандартное отклонение.
\section{Реализация}
Лабораторная работа выполнена на языке Python 3.9 с использованием библиотек numpy, scipy, matplotlib, seaborn.
\section{Результаты}
Более насыщенными цветами выделены полученные результаты, более бледными - теоретические функции распределения и плотности вероятности.
\subsection{Эмпирическая функция распределения}
\begin{figure}[H]
    \centering
    \includegraphics[width = 14 cm]{normalDistr.pdf}
    \caption{Нормальное распределение}
    \label{fig:normalDis}
\end{figure}
\begin{figure}[H]
    \centering
    \includegraphics[width = 15 cm]{koshiDistr.pdf}
    \caption{Распределение Коши}
    \label{fig:koshiDis}
\end{figure}
\begin{figure}[H]
    \centering
    \includegraphics[width = 15 cm]{laplaceDistr.pdf}
    \caption{Распределение Лапласа}
    \label{fig:laplaceDis}
\end{figure}
\begin{figure}[H]
    \centering
    \includegraphics[width = 15 cm]{poissonDistr.pdf}
    \caption{Распределение Пуассона}
    \label{fig:poissonDis}
\end{figure}
\begin{figure}[H]
    \centering
    \includegraphics[width = 15 cm]{uniformDistr.pdf}
    \caption{Равномерное распределение}
    \label{fig:uniformDis}
\end{figure}
\subsection{Ядерные оценки плотности распределения}
\subsubsection{Нормальное распределение}
\begin{figure}[H]
    \centering
    \includegraphics[width = 15 cm]{Normal20numbersKernel.pdf}
    \caption{Нормальное распределение, 20 чисел}
    \label{fig:normalKer20}
\end{figure}
\begin{figure}[H]
    \centering
    \includegraphics[width = 15 cm]{Normal60numbersKernel.pdf}
    \caption{Нормальное распределение, 60 чисел}
    \label{fig:normalKer60}
\end{figure}
\begin{figure}[H]
    \centering
    \includegraphics[width = 15 cm]{Normal100numbersKernel.pdf}
    \caption{Нормальное распределение, 100 чисел}
    \label{fig:normalKer100}
\end{figure}
\subsubsection{Распределение Коши}
\begin{figure}[H]
    \centering
    \includegraphics[width = 15 cm]{Cauchy20numbersKernel.pdf}
    \caption{Распределение Коши, 20 чисел}
    \label{fig:koshiKer20}
\end{figure}
\begin{figure}[H]
    \centering
    \includegraphics[width = 15 cm]{Cauchy60numbersKernel.pdf}
    \caption{Распределение Коши, 60 чисел}
    \label{fig:koshiKer60}
\end{figure}
\begin{figure}[H]
    \centering
    \includegraphics[width = 15 cm]{Cauchy100numbersKernel.pdf}
    \caption{Распределение Коши, 100 чисел}
    \label{fig:koshiKer100}
\end{figure}
\subsubsection{Распределение Лапласа}
\begin{figure}[H]
    \centering
    \includegraphics[width = 15 cm]{Laplace20numbersKernel.pdf}
    \caption{Распределение Лапласа, 20 чисел}
    \label{fig:laplaceKer20}
\end{figure}
\begin{figure}[H]
    \centering
    \includegraphics[width = 15 cm]{Laplace60numbersKernel.pdf}
    \caption{Распределение Лапласа, 60 чисел}
    \label{fig:laplaceKer60}
\end{figure}
\begin{figure}[H]
    \centering
    \includegraphics[width = 15 cm]{Laplace100numbersKernel.pdf}
    \caption{Распределение Лапласа, 100 чисел}
    \label{fig:laplaceKer100}
\end{figure}
\subsubsection{Распределение Пуассона}
\begin{figure}[H]
    \centering
    \includegraphics[width = 15 cm]{Poisson20numbersKernel.pdf}
    \caption{Распределение Пуассона, 20 чисел}
    \label{fig:poissonKer20}
\end{figure}
\begin{figure}[H]
    \centering
    \includegraphics[width = 15 cm]{Poisson60numbersKernel.pdf}
    \caption{Распределение Пуассона, 60 чисел}
    \label{fig:poissonKer60}
\end{figure}
\begin{figure}[H]
    \centering
    \includegraphics[width = 15 cm]{Poisson100numbersKernel.pdf}
    \caption{Распределение Пуассона, 100 чисел}
    \label{fig:poissonKer100}
\end{figure}
\subsubsection{Равномерное распределение}
\begin{figure}[H]
    \centering
    \includegraphics[width = 15 cm]{Uniform20numbersKernel.pdf}
    \caption{Равномерное распределение, 20 чисел}
    \label{fig:uniformKer20}
\end{figure}
\begin{figure}[H]
    \centering
    \includegraphics[width = 15 cm]{Uniform60numbersKernel.pdf}
    \caption{Равномерное распределение, 60 чисел}
    \label{fig:uniformKer60}
\end{figure}
\begin{figure}[H]
    \centering
    \includegraphics[width = 15 cm]{Uniform100numbersKernel.pdf}
    \caption{Равномерное распределение, 100 чисел}
    \label{fig:uniformKer100}
\end{figure}
\section{Обсуждение}
\subsection{Эмпирическая функция и ядерные оценки плотности распределения}
По графикам эмпирической функции распределения видно, что с увеличением мощности выборки растет точность приближения оценки к теоретической функции распределения вероятности. Хуже всего эмпирическая функция распределения приближает теоретическую на равномерном распределении.

По графикам ядерных оценок видно, что при любом выборе параметра $h_n$ увеличение мощности выборки положительно сказывается на точности оценки. 

Для различных распределений больше подходят различные параметры $h_n$: для нормального, пуассоновского и равномерного распределений большую эффективность показывает параметр $h_n$. Для распределений Коши и Лапласа - $\frac{h_n}{2}$.

Также можно сделать вывод, что увеличении коэффициента при параметре $h_n$ ведет к сглаживанию ядерной оценки: количество перегибов у функции уменьшается, при значении параметра $2h_n$ ядерные оценки становятся сложно различимыми у разных распределений.
\section*{Примечание}
С исходным кодом работы и данного отчета можно ознакомиться в репозитории\;\url{https://github.com/Stasychbr/MatStat}
\end{document}
