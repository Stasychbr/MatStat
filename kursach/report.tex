\documentclass[a4paper]{article}
\addtolength{\hoffset}{-2.25cm}
\addtolength{\textwidth}{4.5cm}
\addtolength{\voffset}{-3.25cm}
\addtolength{\textheight}{5cm}
\setlength{\parindent}{15pt}

\usepackage[unicode=true, colorlinks=false, hidelinks]{hyperref}
\usepackage[utf8]{inputenc}
\usepackage[english, russian]{babel}
\usepackage{mathtext}
\usepackage[T2A, TS1]{fontenc}
\usepackage{microtype} % Slightly tweak font spacing for aesthetics
\usepackage{amsthm, amssymb, amsmath, amsfonts, nccmath}
\usepackage{nicefrac}
\usepackage{epstopdf}
\usepackage[export]{adjustbox}
\usepackage{float} % Improved interface for floating objects
\usepackage{graphicx, multicol} % Enhanced support for graphics
\usepackage{pdfrender,xcolor}
\usepackage{breqn}
\usepackage{mathtools}
\usepackage{titling}
\usepackage{bm}
\usepackage{centernot}
\usepackage[cal=boondoxo,calscaled=.96]{mathalpha}
\usepackage{marvosym, wasysym} % More symbols
\usepackage{rotating} % Rotation tools
\usepackage{censor} % Facilities for controlling restricted text
\usepackage{indentfirst}

\DeclareMathOperator{\cov}{cov}
\DeclareMathOperator{\med}{med}
\DeclareMathOperator{\sign}{sign}

\usepackage{array}
\newcolumntype{C}[1]{>{\centering\let\newline\\\arraybackslash\hspace{0pt}}m{#1}}

\usepackage{fancyhdr}
\pagestyle{fancy}
\fancyhead{}\renewcommand{\headrulewidth}{0pt}
\fancyfoot[L]{}
\fancyhead{}
\fancyfoot{}
\fancyfoot[R]{\thepage}
\begin{document}
\large
\begin{center}
    Санкт-Петербургский политехнический университет\\
    Высшая школа прикладной математики и\\вычислительной физики, ИПММ\\
    \vspace{3em}
    Направление подготовки\\
    01.03.02 «Прикладная математика и информатика»\\
    \vspace{10em}
    \Large
    Отчет по курсовой работе \\
    по дисциплине «Математическая статистика»\\
    на тему «Метод главных компонент»
    \vspace{19em}
    \large
\end{center}
Выполнил студент гр. 3630102/80201\\
Кирпиченко С. Р.\\
Руководитель\\
Баженов А. Н.
\vspace{10em}
\begin{center}
    Санкт-Петербург\\
    2021
\end{center}
\thispagestyle{empty}
\newpage
\tableofcontents
\addtocontents{toc}{~\hfill\textbf{Страница}\par}
\newpage
\listoffigures
\addtocontents{lof}{~\hfill\textbf{Страница}\par}
\newpage
\listoftables
\addtocontents{lot}{~\hfill\textbf{Страница}\par}
\newpage
\section{Постановка задачи}

\section{Теория}
\subsection{Постановка задачи метода главных компонент}
В компонентном анализе ищется такое линейное преобразование 
\begin{equation}\label{preobr}
    \widehat{x}=L\widehat{f},
\end{equation}
где $\widehat{x}=(x_1,\hdots,x_d),\;\widehat{f}=(f_1,\hdots,f_d)$ - векторы-столбцы случайных величин и $L=||l_{ij}||$ - квадратная матрица размером $d\times d$, в которой случайные величины $f_1,\hdots,f_d$ некоррелированы и нормированы $\mathbf{E}f_i=0,\; \mathbf{D}f_i=1,\;i=1,\hdots,d$; всегда для простоты предполагается, что $\mathbf{E}x_i=0,\;i=1,\hdots,d$. В этом случае дисперсия выражается как $$\mathbf{D}x_i=l_{i1}^2+\cdots+l_{id}^2,\quad i=1,\hdots, d$$
Следовательно, суммарная дисперсия $\{x_i\}_{i=1}^d$ равна 
\begin{equation}\label{summa}
\sum\limits_{i=1}^d\mathbf{D}x_i=\sum\limits_{i=1}^dl_{i1}^2+\cdots+\sum\limits_{i=1}^dl_{id}^2
\end{equation}
Отыскание представления (\ref{preobr}) эквивалентно определению $d$ таких нормированных линейных комбинаций $y_1,\dots,y_d$ переменных $x_1,\dots,x_d$ (т.е. сумма квадратов коэффициентов равна 1), что для каждого $k=1,\dots,d\;y_k$ имеет наибольшую дисперсию среди всех нормированных линейных комбинаций при условии некоррелированности с предыдущими комбинациями $y_1,\dots,y_{k-1}$. Такие линейные комбинации $y_1,\dots,y_d$ называются \textit{главными компонентами} системы случайных величин $x_1,\dots,x_d$.
\subsection{Алгоритм}
Пусть дана $d\;-$ мерная выборка $\left(X_1,\hdots,X_n\right)$. 
\begin{enumerate}
    \item  Составим матрицу
\begin{equation}
    X=\begin{bmatrix}
    x_1^1&\hdots&x_n^1\\
    x_1^2&\hdots&x_n^2\\
    \hdots&\hdots&\hdots\\
    x_1^d&\hdots&x_n^d\\
    \end{bmatrix}
\end{equation}
    \item Построим ковариационную матрицу
    \begin{equation}
        C=\dfrac{1}{n-1}XX^T.
    \end{equation}
    \item $C$ диагонализуемая, то есть представима в виде
    \begin{equation}
        C = P^T \Lambda P,
    \end{equation}
    где $P^T$ есть ортонормированная матрица, содержащая собственные векторы матрицы $C$, или \textit{главные компоненты}, а $\Lambda$ - диагональная матрица, содержащая соответствующие главным компонентам собственные числа матрицы $C$. Причем, $\lambda_1 \geq \lambda_2 \geq ... \geq \lambda_d > 0$, и $\lambda_i$ есть вклад компоненты $f_i$ в суммарную дисперсию $x_1 ... x_d$, равную в силу \eqref{summa} $\lambda_1 + ... + \lambda_d  = tr(\Lambda)$
    \item Для проекции $X$ на множество главных компонент с индексами $i_1,...i_k$ составим матрицу $P$, столбцами которой будут являться собственые вектора $v_{i_1},...,v_{i_k}$. Тогда проекцией $X$ на множество главных компонент с индексами $i_1,...i_k$ будет являться
    \begin{equation}
        Y=PX.
    \end{equation}
\end{enumerate}

\section{Реализация}
Лабораторная работа выполнена на языке Python в среде PyCharm с использованием библиотек numpy, matplotlib.pyplot. Метод главных компонент был взят из модуля decomposition библиотеки sklearn.

Научным руководителем предоставлено 30 образцов исходных данных \\«Hexane_extr_Kivu_Lake» и 2 научные статьи: \cite{article1}, \cite{article2}. 
\section{Результаты}
\subsection{Примеры образцов}
\begin{figure}[H]
    \centering
    \includegraphics[width=8.0cm]{src/Figure_1.png}
    \includegraphics[width=8.0cm]{src/Figure_3.png}
    \includegraphics[width=8.0cm]{src/Figure_6.png}
    \includegraphics[width=8.0cm]{src/Figure_8.png}
    \caption{Несколько исходных образцов}
    \label{fig:samples}
\end{figure}
Исходные данные включают в себя 30 образцов, 4 из которых для примера изображены на рис. \ref{fig:samples}. Как видим, на всех графиках присутствует шум - две полосы резких пиков. Чтобы убрать данные выбросы из исходных данных, был применен медианный фильтр с размером окна $(10, 4)$. В результате этой операции приведенные на \ref{fig:samples} образцы преобразованы следующим образом:
\begin{figure}[H]
    \centering
    \includegraphics[width=8.0cm]{src/Figure_1_f.png}
    \includegraphics[width=8.0cm]{src/Figure_3_f.png}
    \includegraphics[width=8.0cm]{src/Figure_6_f.png}
    \includegraphics[width=8.0cm]{src/Figure_8_f.png}
    \caption{Сглаженные исходные образцы}
    \label{fig:samples_filtered}
\end{figure}
Как видим, характер и количество экстремумов (исключая описанный шум) было сохранено.

По итогам применения метода главных компонент были получены следующие результаты:
\begin{figure}[H]
    \centering
    \includegraphics[width=15.0cm]{src/hist.png}
    \caption{Гистограмма распределения дисперсий по компонентам}
    \label{fig:hist}
\end{figure}
\begin{figure}[H]
    \centering
    \includegraphics[width=8.0cm]{src/comp1_fig.png}
    \includegraphics[width=8.0cm]{src/comp_1.png}
    \caption{График первой полученной компоненты}
    \label{fig:comp_1}
\end{figure}
\begin{figure}[H]
    \centering
    \includegraphics[width=8.0cm]{src/comp2_fig.png}
    \includegraphics[width=8.0cm]{src/comp_2.png}
    \caption{График второй полученной компоненты}
    \label{fig:comp_2}
\end{figure}
Отметим области максимальной дисперсии на рассмтариваемых главных компонентах для сопоставления их с областями из рис. 7, \cite{article1}.
\begin{figure}[H]
    \centering
    \includegraphics[width=13.0cm]{src/comp1_marked.png}
    \caption{Разбиение главной компоненты 1 на области}
    \label{fig:comp1_marked}
\end{figure}
\begin{figure}[H]
    \centering
    \includegraphics[width=13.0cm]{src/comp2_marked.png}
    \caption{Разбиение главной компоненты 2 на области}
    \label{fig:comp2_marked}
\end{figure}
\begin{table}[H]
    \centering
    \begin{tabular}{|c|c|}
        \hline
        Номер области & Сопоставленное из \cite{article1} вещество\\
        \hline  
        1&Protein-like containing Tryptophan\\
        \hline
        2&Tryptophan and Protein-like Related to Biological\\
        \hline
        3&Humic acid-like\\
        \hline
        4&Marine humic acids\\
        \hline
        5&Humic acid-like\\
        \hline
        6&Humic acid-like\\
        \hline
    \end{tabular}
    \caption{Сопоставление областей с данными}
    \label{tab:marked_regions}
\end{table}
\section{Обсуждение}
\begin{enumerate}
    \item Из гистограммы (рис. \ref{fig:hist}) видно, что первая компонента покрывает примерно 90\% дисперсии исходных данных. Первые две рассмотренные компоненты покрывают почти все 100\% дисперсии;
    \item Максимумы на рис. \ref{fig:comp_1} и рис. \ref{fig:comp_2} обозначают области, в которых дисперсия данных среди  представленных образцов максимальна. Некоторым из этих максимумов удалось сопоставить вещества  из статьи \cite{article1} (табл. \ref{tab:marked_regions}), однако область глобального максимума в первой компоненте не удалось соотнести ни с какой аминокислотой в \cite{article1};
    \item Рельеф первой главной компоненты в целом похож на большую часть исходных данных (например, можно визуально сопоставить рис. \ref{fig:comp_1} и первые три образца из рис. \ref{fig:samples_filtered}).
\end{enumerate}
\section*{Исходный код}
С исходным кодом программы и отчета можно ознакомиться в репозитории \url{https://github.com/Stasychbr/MatStat}.
\begin{thebibliography}{9}
\bibitem[1]{book1} 
 Максимов Ю.Д. Математика. Теория и практика по математической статистике. Конспект-справочник по теории вероятностей : учеб. пособие /
Ю.Д. Максимов; под ред. В.И. Антонова. $-$ СПб. : Изд-во Политехн.
ун-та, 2009. $-$ 395 с. (Математика в политехническом университете).
\bibitem[2]{book2}
Ивченко Г.И., Медведев Ю.И. Математическая статистика: Учебник. — М.: Издательство ЛКИ, 2014. — 352 с. 
\bibitem[3]{book3}
Айвазян, Бухштабер, Енюков, Мешалкин. Прикладная Статистика. Классификация и снижение размерности. - М.: Финансы и статистика, 1989. - 607 с.
\bibitem[4]{article1}
Chen W., Westerhoff P., Leenheer J.A., Booksh K. Fluorescence Excitation-Emission
Matrix Regional Integration to
Quantify Spectra for Dissolved
Organic Matter // Environ. Sci. Technol. 2003, 37, p. 5701-5710
\bibitem[5]{article2}
Semenov P.B., et al. Methane and Dissolved Organic Matter in the Ground Ice Samples from Central Yamal: Implications to Biogeochemical Cycling and Greenhouse Gas Emission. // Geosciences. 2020: 450 с.
\bibitem[6]{article3}
Dramichanin T., Ackovich L.L., Zekovich I., Dramichanin M. D. Detection of Adulterated Honey by Fluorescence
Excitation-Emission Matrices // Hindawi Journal of Spectroscopy
Volume 2018, Article ID 8395212, 6 p.
\end{thebibliography}
\end{document}
